\documentclass[12pt,a4paper]{article}
\usepackage[UTF8]{ctex}
\usepackage{geometry}
\geometry{left=2.6cm,right=2.6cm,top=2.8cm,bottom=2.8cm}

\usepackage{setspace}
\onehalfspacing
\setlength{\parindent}{2em}

\usepackage{xcolor}
\definecolor{darkgreen}{RGB}{34,102,85}

\usepackage{titlesec}
\titleformat{\section}{\Large\bfseries\color{darkgreen}}{\thesection}{1em}{}[\titlerule]
\titleformat{\subsection}{\large\bfseries}{\thesubsection}{1em}{}
\titlespacing*{\section}{0pt}{3ex}{2ex}
\titlespacing*{\subsection}{0pt}{2ex}{1.2ex}

\usepackage{array}
\usepackage{graphicx}
\usepackage{booktabs}
\usepackage{float}
\usepackage{caption}
\captionsetup{font=small,labelfont=bf}

\usepackage{listings}
\definecolor{codebg}{RGB}{248,249,250}        % 极浅冷灰背景
\definecolor{codeframe}{RGB}{215,218,223}     % 柔和灰边框
\definecolor{codecomment}{RGB}{106,115,125}   % 注释灰,偏冷色
\definecolor{codekeyword}{RGB}{80,130,200}    % 浅蓝关键字,清新自然
\definecolor{codestring}{RGB}{200,120,100}    % 浅珊瑚红字符串,柔和区分

\lstset{
  backgroundcolor=\color{codebg},             % 背景
  basicstyle=\ttfamily\small,                 % 等宽字体
  keywordstyle=\bfseries\color{codekeyword},  % 关键字黑色加粗
  commentstyle=\itshape\color{codecomment},   % 注释斜体灰色
  stringstyle=\color{codestring},             % 字符串黑色
  frame=single,                               % 单线边框
  rulecolor=\color{codeframe},                % 边框颜色
  numbers=left,                               % 行号在左
  numberstyle=\tiny\color{codecomment},       % 行号灰色小号
  breaklines=true,                            % 自动换行
  tabsize=2,
  captionpos=b,
  xleftmargin=1em,                            % 左边距
  xrightmargin=1em                            % 右边距
}

\usepackage{hyperref}
\hypersetup{colorlinks=true,linkcolor=black,urlcolor=darkgreen}

% 页眉风格:左上标题,右上学号+姓名
\usepackage{fancyhdr}
\pagestyle{fancy}
\fancyhf{}
\fancyhead[L]{数据库系统实验报告}
\fancyhead[R]{09023402-金俊贤}
\renewcommand{\headrulewidth}{0.4pt}

% 封面宏
\newcommand{\reportTitle}[3]{%
\begin{titlepage}
\vspace*{2cm}
\begin{center}
{\Huge\bfseries #1 \par}
\vspace{1cm}
{\Large\color{darkgreen}#2 \par}
\vspace{3cm}
\renewcommand{\arraystretch}{1.5}
\begin{tabular}{rl}
课程名称: & #3 \\
学号: & 09023402 \\
姓名: & 金俊贤 \\
时间: & \today
\end{tabular}
\end{center}
\end{titlepage}
}

% ====================================================================================================

\begin{document}

\reportTitle{数据库系统实验报告}{—— 网咖会员与消费追踪系统 ——}{数据库系统原理}

\tableofcontents
\newpage

% ====================================================================================================

\section{实验场景介绍}
本实验以网吧作为应用场景,设计并实现了“网咖会员与消费追踪系统”的数据库。
系统用于管理网吧的会员信息、上机消费记录、充值记录以及座位占用情况,实现日常运营的数据化管理。
系统主要服务于两类用户:  
\begin{itemize}
    \item \textbf{顾客}:通过会员卡或账号登录,能够完成充值、上机及查询个人消费记录等操作。  
    \item \textbf{管理员}:负责维护会员信息、管理座位状态、统计收入和生成账单。  
\end{itemize}

系统设计目标是实现对网吧运营的全面管理,包括实时记录顾客的上机起止时间及相应消费金额,确保数据准确可靠;
自动计算消费费用并生成账单,减少人工操作,提高管理效率;
对座位状态进行动态管理,包括空闲、使用中及维修等情况,实现资源的合理调度;
同时支持按用户、日期和座位的多维度统计与查询,为经营决策提供可靠的数据依据。

\section{数据库设计}
\subsection{数据流图}
本系统在数据库设计中采用 Level-0 数据流图(图 \ref{fig:dataflow}),用于描述系统与外部实体之间的主要数据交互关系。
系统以“网咖会员与消费追踪系统”为核心处理过程,外部实体包括顾客和管理员。
顾客通过系统完成会员办理、账户充值、上机消费及信息查询等操作,管理员主要负责座位管理和信息查询。
系统在业务处理过程中对用户、座位、充值及消费等数据进行读写,相关数据统一存储于数据库中。
由于系统规模较小,且后续章节中已对各项业务功能的实现过程进行了详细说明,因此本实验采用 Level-0 数据流图对系统进行整体描述。

\begin{figure}[htbp]
    \centering
    \includegraphics[width=0.8\textwidth]{../img/DataFlow.png}
    \caption{系统 Level-0 数据流图}
    \label{fig:dataflow}
\end{figure}

\subsection{概念模式 E-R 图设计}
本系统的概念模式采用实体-关系(E-R)建模方法,设计了四类核心实体:用户(User)、座位(Seat)、充值(Recharge)及上机消费(Consumption),以及它们之间的关联关系。
系统 E-R 图如图 \ref{fig:ER} 所示。

\begin{figure}[htbp]
    \centering
    \includegraphics[width=0.8\textwidth]{../img/ER.png}
    \caption{系统概念模式 E-R 图}
    \label{fig:ER}
\end{figure}

系统的核心实体及其主要属性如下:
\begin{itemize}
    \item \textbf{User}:存储会员信息,包括用户编号(\texttt{user\_id})、姓名(\texttt{name})、会员卡号(\texttt{membership\_card})、联系方式(\texttt{phone})及账户余额(\texttt{balance})。
    \item \textbf{Seat}:记录座位编号(\texttt{seat\_id})、位置(\texttt{location})及状态(\texttt{status})。
    \item \textbf{Recharge}:保存用户的充值记录,包括充值编号(\texttt{recharge\_id})、用户编号(\texttt{user\_id})、充值金额(\texttt{amount})及充值时间(\texttt{recharge\_time})。
    \item \textbf{Consumption}:记录用户上机消费情况,包括消费编号(\texttt{consumption\_id})、用户编号(\texttt{user\_id})、座位编号(\texttt{seat\_id})、上机开始时间(\texttt{start\_time})、结束时间(\texttt{end\_time})及消费金额(\texttt{fee})。
\end{itemize}

实体之间的关系设计如下:
\begin{itemize}
    \item \textbf{注册}:用户与充值之间存在一对一关系。
    \item \textbf{充值}:用户与充值之间存在一对多关系,即每个用户可对应多条充值记录。
    \item \textbf{上机消费}:用户与消费记录之间存在一对多关系,每个用户可对应多条消费记录;每条消费记录关联唯一座位。
    \item \textbf{座位管理}:座位状态随用户上机或下机而更新,确保座位资源合理调度。
\end{itemize}

\subsection{数据模式设计}
根据系统功能需求,系统主要涉及用户、座位、充值和上机消费等数据内容,因此在数据模式设计中分别设置了 User 表、Seat 表、Recharge 表和 Consumption 表。
各数据表的具体字段设计如下:

\begin{itemize}
    \item \textbf{User Table}
    \begin{itemize}
        \item \texttt{user\_id INT PRIMARY KEY}:用户唯一标识
        \item \texttt{name VARCHAR(50)}:用户名
        \item \texttt{membership\_card VARCHAR(20)}:会员卡号
        \item \texttt{phone VARCHAR(15)}:联系方式
        \item \texttt{balance DECIMAL(8,2)}:当前余额
    \end{itemize}
    \item \textbf{Seat Table}
    \begin{itemize}
        \item \texttt{seat\_id INT PRIMARY KEY}:座位编号
        \item \texttt{location VARCHAR(50)}:座位位置(如 A1、B3)
        \item \texttt{status VARCHAR(10)}:座位状态(空闲 / 使用中 / 维修)
    \end{itemize}
    \item \textbf{Recharge Table}
    \begin{itemize}
        \item \texttt{recharge\_id INT PRIMARY KEY}:充值记录编号
        \item \texttt{user\_id INT}:外键,关联 \texttt{User.user\_id}
        \item \texttt{amount DECIMAL(8,2)}:充值金额
        \item \texttt{recharge\_time DATETIME}:充值时间
    \end{itemize}
    \item \textbf{Consumption Table}
    \begin{itemize}
        \item \texttt{consumption\_id INT PRIMARY KEY}:消费记录编号
        \item \texttt{user\_id INT}:外键,关联 \texttt{User.user\_id}
        \item \texttt{seat\_id INT}:外键,关联 \texttt{Seat.seat\_id}
        \item \texttt{start\_time DATETIME}:上机开始时间
        \item \texttt{end\_time DATETIME}:上机结束时间
        \item \texttt{fee DECIMAL(8,2)}:本次消费费用
    \end{itemize}
\end{itemize}

其中,User 表用于保存会员的基本信息及账户余额,为后续的充值和上机消费操作提供数据支持;
Recharge 表记录用户的每一次充值情况,便于按照用户或时间对充值记录进行查询;
Consumption 表用于保存用户上机过程中的开始时间、结束时间及对应费用,从而实现对上机消费过程的管理;
Seat 表主要记录网吧座位的编号、位置及当前使用状态,并在用户上机和下机过程中进行相应更新。
通过上述数据表的配合使用,系统能够完成会员管理、充值管理、上机消费管理以及相关的查询与统计功能。

\section{系统功能介绍}
本系统围绕网吧日常运营中的核心业务流程进行设计,主要实现会员管理、充值管理、上机消费管理以及座位管理等功能。
系统以数据库为核心,对用户信息、资金变动和座位使用情况进行统一管理,保证数据的一致性和可追溯性。

\subsection{会员管理功能}
系统支持顾客办理会员并建立用户档案,统一存储会员的基本信息及账户余额。
管理员可对会员信息进行维护与查询,系统能够根据用户编号快速定位对应的会员数据,为后续的充值和消费操作提供基础支持。

\subsection{充值管理功能}
顾客可通过系统进行账户充值。每次充值操作都会生成一条充值记录,同时自动更新用户账户余额。
充值记录按照时间顺序保存,便于后续按用户或时间范围进行查询和统计,确保充值数据的完整性和可追溯性。

\subsection{上机消费管理功能}
系统对顾客的上机行为进行完整记录,包括上机开始时间、结束时间以及对应的消费费用。
消费结束后,系统根据使用时长计算费用,并从用户账户余额中扣除相应金额,同时生成消费记录,从而实现上机消费过程的自动化管理。

\subsection{座位管理功能}
系统对网吧座位信息进行集中管理,记录每个座位的编号、位置及当前使用状态。
管理员可根据实际情况调整座位状态,如设置为使用中、空闲或维修状态。在用户上机和下机过程中,系统会同步更新座位状态,避免座位资源冲突。

\subsection{查询与统计功能}
系统支持对会员信息、充值记录及消费记录的查询操作,可按照用户、时间或座位等条件进行筛选。
同时,管理员可基于现有数据进行统计分析,为日常管理和运营决策提供数据支持。

\section{核心代码}
% --------------------------------------------------------------

\begin{lstlisting}[language=SQL, caption=]

\end{lstlisting}



\section{效果展现}

\section{实验总结}

\end{document}