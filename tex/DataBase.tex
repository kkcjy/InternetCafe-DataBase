\documentclass[12pt,a4paper]{article}
\usepackage[UTF8]{ctex}
\usepackage{geometry}
\geometry{left=2.6cm,right=2.6cm,top=2.8cm,bottom=2.8cm}

\usepackage{setspace}
\onehalfspacing
\setlength{\parindent}{2em}

\usepackage{xcolor}
\definecolor{darkgreen}{RGB}{34,102,85}

\usepackage{titlesec}
\titleformat{\section}{\Large\bfseries\color{darkgreen}}{\thesection}{1em}{}[\titlerule]
\titleformat{\subsection}{\large\bfseries}{\thesubsection}{1em}{}
\titlespacing*{\section}{0pt}{3ex}{2ex}
\titlespacing*{\subsection}{0pt}{2ex}{1.2ex}

\usepackage{array}
\usepackage{graphicx}
\usepackage{booktabs}
\usepackage{float}
\usepackage{caption}
\captionsetup{font=small,labelfont=bf}

\usepackage{listings}
\definecolor{codebg}{RGB}{248,249,250}        % 极浅冷灰背景
\definecolor{codeframe}{RGB}{215,218,223}     % 柔和灰边框
\definecolor{codecomment}{RGB}{106,115,125}   % 注释灰,偏冷色
\definecolor{codekeyword}{RGB}{80,130,200}    % 浅蓝关键字,清新自然
\definecolor{codestring}{RGB}{200,120,100}    % 浅珊瑚红字符串,柔和区分

\lstset{
  backgroundcolor=\color{codebg},             % 背景
  basicstyle=\ttfamily\small,                 % 等宽字体
  keywordstyle=\bfseries\color{codekeyword},  % 关键字黑色加粗
  commentstyle=\itshape\color{codecomment},   % 注释斜体灰色
  stringstyle=\color{codestring},             % 字符串黑色
  frame=single,                               % 单线边框
  rulecolor=\color{codeframe},                % 边框颜色
  numbers=left,                               % 行号在左
  numberstyle=\tiny\color{codecomment},       % 行号灰色小号
  breaklines=true,                            % 自动换行
  tabsize=2,
  captionpos=b,
  xleftmargin=1em,                            % 左边距
  xrightmargin=1em                            % 右边距
}

\usepackage{hyperref}
\hypersetup{colorlinks=true,linkcolor=black,urlcolor=darkgreen}

% 页眉风格:左上标题,右上学号+姓名
\usepackage{fancyhdr}
\pagestyle{fancy}
\fancyhf{}
\fancyhead[L]{数据库系统实验报告}
\fancyhead[R]{09023402-金俊贤}
\renewcommand{\headrulewidth}{0.4pt}

% 封面宏
\newcommand{\reportTitle}[3]{%
\begin{titlepage}
\vspace*{2cm}
\begin{center}
{\Huge\bfseries #1 \par}
\vspace{1cm}
{\Large\color{darkgreen}#2 \par}
\vspace{3cm}
\renewcommand{\arraystretch}{1.5}
\begin{tabular}{rl}
课程名称: & #3 \\
学号: & 09023402 \\
姓名: & 金俊贤 \\
链接: & \href{https://github.com/kkcjy/InternetCafe-DataBase.git}{InternetCafe-DataBase} \\
时间: & \today
\end{tabular}
\end{center}
\end{titlepage}
}

% ====================================================================================================

\begin{document}

\reportTitle{数据库系统实验报告}{—— 网咖会员与消费追踪系统 ——}{数据库系统原理}

\tableofcontents
\newpage

% ====================================================================================================

\section{实验场景介绍}
本实验以网吧作为应用场景,设计并实现了“网咖会员与消费追踪系统”的数据库。
系统用于管理网吧的会员信息、上机消费记录、充值记录以及座位占用情况,实现日常运营的数据化管理。
系统主要服务于两类用户:  
\begin{itemize}
    \item \textbf{顾客}:通过会员卡或账号登录,能够完成充值、上机及查询个人消费记录等操作。  
    \item \textbf{管理员}:负责维护会员信息、管理座位状态、统计收入和生成账单。  
\end{itemize}

系统设计目标是实现对网吧运营的全面管理,包括实时记录顾客的上机起止时间及相应消费金额,确保数据准确可靠;
自动计算消费费用并生成账单,减少人工操作,提高管理效率;
对座位状态进行动态管理,包括空闲、使用中及维修等情况,实现资源的合理调度;
同时支持按用户、日期和座位的多维度统计与查询,为经营决策提供可靠的数据依据。

\section{数据库设计}
\subsection{数据流图}
本系统在数据库设计中采用 Level-0 数据流图(图 \ref{fig:dataflow}),用于描述系统与外部实体之间的主要数据交互关系。
系统以“网咖会员与消费追踪系统”为核心处理过程,外部实体包括顾客和管理员。
顾客通过系统完成会员办理、账户充值、上机消费及信息查询等操作,管理员主要负责座位管理和信息查询。
系统在业务处理过程中对用户、座位、充值及消费等数据进行读写,相关数据统一存储于数据库中。
由于系统规模较小,且后续章节中已对各项业务功能的实现过程进行了详细说明,因此本实验采用 Level-0 数据流图对系统进行整体描述。

\begin{figure}[htbp]
    \centering
    \includegraphics[width=0.8\textwidth]{../img/DataFlow.png}
    \caption{系统 Level-0 数据流图}
    \label{fig:dataflow}
\end{figure}

\subsection{概念模式 E-R 图设计}
本系统的概念模式采用实体-关系(E-R)建模方法,设计了四类核心实体:用户(User)、座位(Seat)、充值(Recharge)及上机消费(Consumption),以及它们之间的关联关系。
系统 E-R 图如图 \ref{fig:ER} 所示。

\begin{figure}[htbp]
    \centering
    \includegraphics[width=0.8\textwidth]{../img/ER.png}
    \caption{系统概念模式 E-R 图}
    \label{fig:ER}
\end{figure}

系统的核心实体及其主要属性如下:
\begin{itemize}
    \item \textbf{User}:存储会员信息,包括用户编号(\texttt{user\_id})、姓名(\texttt{name})、会员卡号(\texttt{membership\_card})、联系方式(\texttt{phone})及账户余额(\texttt{balance})。
    \item \textbf{Seat}:记录座位编号(\texttt{seat\_id})、位置(\texttt{location})及状态(\texttt{status})。
    \item \textbf{Recharge}:保存用户的充值记录,包括充值编号(\texttt{recharge\_id})、用户编号(\texttt{user\_id})、充值金额(\texttt{amount})及充值时间(\texttt{recharge\_time})。
    \item \textbf{Consumption}:记录用户上机消费情况,包括消费编号(\texttt{consumption\_id})、用户编号(\texttt{user\_id})、座位编号(\texttt{seat\_id})、上机开始时间(\texttt{start\_time})、结束时间(\texttt{end\_time})及消费金额(\texttt{fee})。
\end{itemize}

实体之间的关系设计如下:
\begin{itemize}
    \item \textbf{注册}:用户与充值之间存在一对一关系。
    \item \textbf{充值}:用户与充值之间存在一对多关系,即每个用户可对应多条充值记录。
    \item \textbf{上机消费}:用户与消费记录之间存在一对多关系,每个用户可对应多条消费记录;每条消费记录关联唯一座位。
    \item \textbf{座位管理}:座位状态随用户上机或下机而更新,确保座位资源合理调度。
\end{itemize}

\subsection{数据模式设计}
根据系统功能需求,系统主要涉及用户、座位、充值和上机消费等数据内容,因此在数据模式设计中分别设置了 User 表、Seat 表、Recharge 表和 Consumption 表。
各数据表的具体字段设计如下:

\begin{itemize}
    \item \textbf{User Table}
    \begin{itemize}
        \item \texttt{user\_id INT PRIMARY KEY}:用户唯一标识
        \item \texttt{name VARCHAR(50)}:用户名
        \item \texttt{membership\_card VARCHAR(20)}:会员卡号
        \item \texttt{phone VARCHAR(15)}:联系方式
        \item \texttt{balance DECIMAL(8,2)}:当前余额
    \end{itemize}
    \item \textbf{Seat Table}
    \begin{itemize}
        \item \texttt{seat\_id INT PRIMARY KEY}:座位编号
        \item \texttt{location VARCHAR(50)}:座位位置(如 A1、B3)
        \item \texttt{status VARCHAR(10)}:座位状态(空闲 / 使用中 / 维修)
    \end{itemize}
    \item \textbf{Recharge Table}
    \begin{itemize}
        \item \texttt{recharge\_id INT PRIMARY KEY}:充值记录编号
        \item \texttt{user\_id INT}:外键,关联 \texttt{User.user\_id}
        \item \texttt{amount DECIMAL(8,2)}:充值金额
        \item \texttt{recharge\_time DATETIME}:充值时间
    \end{itemize}
    \item \textbf{Consumption Table}
    \begin{itemize}
        \item \texttt{consumption\_id INT PRIMARY KEY}:消费记录编号
        \item \texttt{user\_id INT}:外键,关联 \texttt{User.user\_id}
        \item \texttt{seat\_id INT}:外键,关联 \texttt{Seat.seat\_id}
        \item \texttt{start\_time DATETIME}:上机开始时间
        \item \texttt{end\_time DATETIME}:上机结束时间
        \item \texttt{fee DECIMAL(8,2)}:本次消费费用
    \end{itemize}
\end{itemize}

其中,User 表用于保存会员的基本信息及账户余额,为后续的充值和上机消费操作提供数据支持;
Recharge 表记录用户的每一次充值情况,便于按照用户或时间对充值记录进行查询;
Consumption 表用于保存用户上机过程中的开始时间、结束时间及对应费用,从而实现对上机消费过程的管理;
Seat 表主要记录网吧座位的编号、位置及当前使用状态,并在用户上机和下机过程中进行相应更新。
通过上述数据表的配合使用,系统能够完成会员管理、充值管理、上机消费管理以及相关的查询与统计功能。

\section{系统功能介绍}
本系统围绕网吧日常运营中的核心业务流程进行设计,主要实现会员管理、充值管理、上机消费管理以及座位管理等功能。
系统以数据库为核心,对用户信息、资金变动和座位使用情况进行统一管理,保证数据的一致性和可追溯性。

\subsection{会员管理功能}
系统支持顾客办理会员并建立用户档案,统一存储会员的基本信息及账户余额。
管理员可对会员信息进行维护与查询,系统能够根据用户编号快速定位对应的会员数据,为后续的充值和消费操作提供基础支持。

\subsection{充值管理功能}
顾客可通过系统进行账户充值。每次充值操作都会生成一条充值记录,同时自动更新用户账户余额。
充值记录按照时间顺序保存,便于后续按用户或时间范围进行查询和统计,确保充值数据的完整性和可追溯性。

\subsection{上机消费管理功能}
系统对顾客的上机行为进行完整记录,包括上机开始时间、结束时间以及对应的消费费用。
消费结束后,系统根据使用时长计算费用,并从用户账户余额中扣除相应金额,同时生成消费记录,从而实现上机消费过程的自动化管理。

\subsection{座位管理功能}
系统对网吧座位信息进行集中管理,记录每个座位的编号、位置及当前使用状态。
管理员可根据实际情况调整座位状态,如设置为使用中、空闲或维修状态。在用户上机和下机过程中,系统会同步更新座位状态,避免座位资源冲突。

\subsection{查询与统计功能}
系统支持对会员信息、充值记录及消费记录的查询操作,可按照用户、时间或座位等条件进行筛选。
同时,管理员可基于现有数据进行统计分析,为日常管理和运营决策提供数据支持。

\section{核心代码}
本系统的核心代码主要由数据表定义、业务存储过程以及功能测试脚本三部分组成。
系统共设计了四张基础数据表,分别用于存储会员信息、座位信息、充值记录和上机消费记录,构成网咖业务运行所需的核心数据结构。
在此基础上,针对会员办理、账户充值、上机消费、消费结算、座位状态维护以及信息查询等业务需求,编写了多组存储过程,将主要业务逻辑封装在数据库层完成。
此外,通过统一的测试脚本对各项功能进行串联调用,模拟实际使用场景下的完整业务流程,对系统功能进行验证。

\subsection{数据表定义}
系统根据业务需求设计了四张核心数据表,分别用于存储会员信息、座位信息、充值记录以及上机消费记录。
各数据表之间通过外键进行关联,确保数据完整性和一致性。
\begin{lstlisting}[language=SQL,caption={系统核心数据表定义}]
CREATE TABLE IF NOT EXISTS User (
    user_id INT PRIMARY KEY AUTO_INCREMENT,
    name VARCHAR(50) NOT NULL,
    membership_card VARCHAR(20) UNIQUE NOT NULL,
    phone VARCHAR(15),
    balance DECIMAL(8,2) DEFAULT 0.00
);

CREATE TABLE IF NOT EXISTS Seat (
    seat_id INT PRIMARY KEY AUTO_INCREMENT,
    location VARCHAR(50) NOT NULL,
    status VARCHAR(10) NOT NULL DEFAULT '空闲'
);

CREATE TABLE IF NOT EXISTS Recharge (
    recharge_id INT PRIMARY KEY AUTO_INCREMENT,
    user_id INT NOT NULL,
    amount DECIMAL(8,2) NOT NULL,
    recharge_time DATETIME NOT NULL DEFAULT CURRENT_TIMESTAMP,
    FOREIGN KEY (user_id) REFERENCES User(user_id)
);

CREATE TABLE IF NOT EXISTS Consumption (
    consumption_id INT PRIMARY KEY AUTO_INCREMENT,
    user_id INT NOT NULL,
    seat_id INT NOT NULL,
    start_time DATETIME NOT NULL,
    end_time DATETIME NULL,
    fee DECIMAL(8,2) NULL,
    FOREIGN KEY (user_id) REFERENCES User(user_id),
    FOREIGN KEY (seat_id) REFERENCES Seat(seat_id)
);
\end{lstlisting}

\subsection{会员办理与充值管理}
会员办理功能用于实现新用户的注册操作。
系统在注册过程中首先对会员卡号进行唯一性检查,防止重复办理会员卡;若卡号不存在,则创建新的用户记录,并将账户初始余额设置为 0。
\begin{lstlisting}[language=SQL,caption={会员办理功能}]
CREATE PROCEDURE AddUser(
    IN p_name VARCHAR(50),
    IN p_membership_card VARCHAR(20),
    IN p_phone VARCHAR(15)
)
BEGIN
    DECLARE v_count INT;
    -- 检查是否已有相同会员卡号
    SELECT COUNT(*) INTO v_count
    FROM User
    WHERE membership_card = p_membership_card;

    IF v_count > 0 THEN
        SIGNAL SQLSTATE '45000'
        SET MESSAGE_TEXT = '该会员卡号已存在';
    ELSE
        INSERT INTO User(name, membership_card, phone, balance)
        VALUES (p_name, p_membership_card, p_phone, 0.00);
    END IF;
END;
\end{lstlisting}

充值管理功能用于为已注册会员进行账户充值操作。
系统根据用户姓名定位对应的会员账户,在更新账户余额的同时生成一条充值记录,并自动记录充值时间。
\begin{lstlisting}[language=SQL,caption={账户充值功能}]
CREATE PROCEDURE RechargeAccount(
    IN p_name VARCHAR(50),
    IN p_amount DECIMAL(8,2)
)
BEGIN
    -- 更新用户余额
    UPDATE User
    SET balance = balance + p_amount
    WHERE name = p_name;
    -- 插入充值记录
    INSERT INTO Recharge(user_id, amount)
    SELECT user_id, p_amount
    FROM User
    WHERE name = p_name;
END;
\end{lstlisting}

\subsection{上机消费与结算管理}
上机消费与结算管理模块用于记录会员的上机行为,并对消费过程进行完整管理。
当用户开始上机时,系统根据会员姓名获取对应的用户编号,并检查目标座位的当前状态,仅在座位处于空闲状态时才允许创建消费记录,同时将座位状态更新为“使用中”。
在消费结束后,通过结算功能对本次上机费用进行处理,系统从用户账户余额中扣除相应金额,并将对应座位状态恢复为空闲,从而完成一次完整的上机消费流程。
\begin{lstlisting}[language=SQL,caption={上机消费记录功能}]
CREATE PROCEDURE RecordConsumption(
    IN p_name VARCHAR(50),
    IN p_seat_id INT,
    IN p_start DATETIME
)
BEGIN
    DECLARE v_user_id INT;
    DECLARE v_seat_status VARCHAR(10);
    -- 获取用户ID
    SELECT user_id INTO v_user_id
    FROM User
    WHERE name = p_name;
    -- 获取座位状态
    SELECT status INTO v_seat_status
    FROM Seat
    WHERE seat_id = p_seat_id;

    -- 判断座位是否空闲
    IF v_seat_status <> '空闲' THEN
        SIGNAL SQLSTATE '45000'
        SET MESSAGE_TEXT = '该座位当前不可用';
    ELSE
        -- 插入消费记录
        INSERT INTO Consumption(user_id, seat_id, start_time, end_time, fee)
        VALUES (v_user_id, p_seat_id, p_start, NULL, NULL);
        -- 更新座位状态为使用中
        UPDATE Seat
        SET status = '使用中'
        WHERE seat_id = p_seat_id;
    END IF;
END;
\end{lstlisting}

\begin{lstlisting}[language=SQL,caption={消费结算功能}]
CREATE PROCEDURE SettleConsumption(
    IN p_name VARCHAR(50),
    IN p_seat_id INT,
    IN p_end DATETIME,
    IN p_fee DECIMAL(8,2)
)
BEGIN
    DECLARE v_user_id INT;
    -- 获取用户ID
    SELECT user_id INTO v_user_id
    FROM User
    WHERE name = p_name;
    -- 更新消费记录:结束时间和费用
    UPDATE Consumption
    SET end_time = p_end,
        fee = p_fee
    WHERE user_id = v_user_id
      AND seat_id = p_seat_id
      AND end_time IS NULL;
    -- 更新座位状态为空闲
    UPDATE Seat
    SET status = '空闲'
    WHERE seat_id = p_seat_id;
END;
\end{lstlisting}

\subsection{座位管理与管理员统计}
座位管理与管理员统计模块主要用于维护座位状态和进行系统汇总分析。
在座位管理方面,管理员可根据座位位置调整座位状态,包括空闲、使用中或维修状态,从而对座位资源进行动态管理和调度,确保座位在维护或异常情况下不会被误占用。
管理员汇总查询功能提供系统级统计能力,能够一次性查看所有会员的总充值、总消费及当前余额,同时展示所有座位的状态信息,为运营管理和决策提供全面的数据支持。
\begin{lstlisting}[language=SQL,caption={座位状态管理功能}]
DELIMITER //
CREATE PROCEDURE UpdateSeatStatus(
    IN p_location VARCHAR(50),
    IN p_status VARCHAR(10)
)
BEGIN
    UPDATE Seat
    SET status = p_status
    WHERE location = p_location;
END;
//
DELIMITER ;
\end{lstlisting}

\begin{lstlisting}[language=SQL,caption={管理员汇总查询功能}]
DELIMITER //
CREATE PROCEDURE AdminQuerySummary()
BEGIN
    -- 查询所有用户的总充值、总消费、当前余额
    SELECT 
        u.name,
        COALESCE(SUM(r.amount), 0) AS total_recharge,
        COALESCE(SUM(c.fee), 0) AS total_consumption,
        u.balance
    FROM User u
    LEFT JOIN Recharge r ON u.user_id = r.user_id
    LEFT JOIN Consumption c ON u.user_id = c.user_id
    GROUP BY u.user_id, u.name, u.balance;
    -- 查询所有座位状态
    SELECT seat_id, location, status
    FROM Seat;
END;
//
DELIMITER ;
\end{lstlisting}

\section{效果展现}
本小节通过执行测试脚本对系统各项功能进行验证,模拟实际使用场景下的完整业务流程,所有测试结果均通过调用$CALL AdminQuerySummary();$获取。

\subsection{座位状态初始化}
首先初始化座位数据,并通过存储过程设置部分座位状态为维修状态。
\begin{lstlisting}[language=SQL,caption={座位状态初始化}]
-- 初始化座位,将 B2 设置为维修状态
INSERT INTO Seat(location, status) VALUES
('A1', '空闲'),
('A2', '空闲'),
('B1', '空闲'),
('B2', '空闲');
CALL UpdateSeatStatus('B2', '维修');
SELECT '座位初始化完成,A1、A2、B1 为空闲状态,B2 为维修状态' AS info;
\end{lstlisting}

\subsection{会员办理与充值管理}
随后,进行会员注册操作,并进行会员卡重复的异常测试,再分别进行账户充值。
\textbf{由于 SQL 脚本执行时控制台输出信息混乱,本次测试将部分执行结果保存到文件中,并截取关键部分进行展示,完整测试脚本及执行结果可在代码仓库中验证。}
\begin{lstlisting}[language=SQL,caption={会员办理与充值管理}]
-- 会员注册 
CALL AddUser('张三', 'M001', '13800000001');
CALL AddUser('李四', 'M002', '13800000002');
CALL AddUser('王五', 'M003', '13800000003');

-- (异常测试:会员卡号重复)
-- CALL AddUser('赵六', 'M001', '13800000004');

-- 账户充值 
CALL RechargeAccount('张三', 100.00);
CALL RechargeAccount('李四', 50.00);
CALL RechargeAccount('王五', 200.00);
\end{lstlisting}

处理结果如图 \ref{fig:test1} 所示:会员办理与充值功能运行正常,同时系统能够检测重复的会员卡号并提示错误。
\begin{figure}[htbp]
    \centering
    \begin{minipage}{0.4\textwidth}
        \centering
        \includegraphics[width=\textwidth]{../img/test1.1.png}
        \caption{会员办理和充值管理测试结果}
    \end{minipage}
    \begin{minipage}{0.4\textwidth}
        \centering
        \includegraphics[width=\textwidth]{../img/test1.2.png}
        \caption{会员办理异常测试结果}
    \end{minipage}
    \caption{会员办理与充值管理}
    \label{fig:test1}
\end{figure}

\subsection{上机消费与结算管理}
上机消费记录的创建与结算操作如图所示,需要注意,这两步操作是分开独立进行的,同时测试中包括对维修座位及使用中座位的异常处理,系统能够正确阻止非法操作。
\begin{lstlisting}[language=SQL,caption={上机消费与结算管理}]
-- 上机消费
CALL RecordConsumption('张三', 1, '2025-12-23 10:00:00', '2025-12-23 12:00:00', 20.00);
CALL RecordConsumption('李四', 2, '2025-12-23 11:00:00', '2025-12-23 13:30:00', 35.00);
CALL RecordConsumption('王五', 3, '2025-12-23 14:00:00', '2025-12-23 16:00:00', 25.00);

-- (异常测试:维修座位不可用)
-- CALL RecordConsumption('张三', 4, '2025-12-23 09:00:00', '2025-12-23 10:00:00', 10.00);
-- (异常测试:使用中座位不可用)
-- CALL RecordConsumption('赵六', 2, '2025-12-23 10:30:00', '2025-12-23 11:30:00', 10.00);

-- 消费结算 
CALL SettleConsumption('张三', 1, 20.00);
CALL SettleConsumption('李四', 2, 35.00);
CALL SettleConsumption('王五', 3, 25.00);
\end{lstlisting}

处理结果如图 \ref{fig:test2} 所示:上机消费与结算功能运行正常,同时系统能够检测维修座位和使用中座位的异常情况并提示错误。
\begin{figure}[htbp]
    \centering
    \begin{minipage}{0.4\textwidth}
        \centering
        \includegraphics[width=\textwidth]{../img/test2.1.png}
        \caption{上机消费测试结果}
    \end{minipage}
    \begin{minipage}{0.4\textwidth}
        \centering
        \includegraphics[width=\textwidth]{../img/test2.2.png}
        \caption{结算管理测试结果}
    \end{minipage}
    \begin{minipage}{0.4\textwidth}
        \centering
        \includegraphics[width=\textwidth]{../img/test2.3.png}
        \caption{选用维修座位异常测试结果}
    \end{minipage}
    \begin{minipage}{0.4\textwidth}
        \centering
        \includegraphics[width=\textwidth]{../img/test2.4.png}
        \caption{选用使用中座位异常测试结果}
    \end{minipage}
    \caption{上机消费与结算管理}
    \label{fig:test2}
\end{figure}

\section{项目构建}

\begin{figure}[htbp]
    \centering
    \begin{minipage}{0.32\textwidth}
        \centering
        \includegraphics[width=\textwidth]{../img/user.png}
        \caption{用户界面}
    \end{minipage}
    \begin{minipage}{0.32\textwidth}
        \centering
        \includegraphics[width=\textwidth]{../img/admin.png}
        \caption{管理员界面}
    \end{minipage}
    \caption{项目页面}
    \label{fig:ui}
\end{figure}

基于前述设计思路及 SQL 测试脚本,本项目实现了一个简易的数据库管理系统。
系统提供用户模式和管理员模式两种操作界面:用户模式支持会员注册、账户充值、上机消费及结算功能;管理员模式支持用户查询、座位状态管理及系统数据汇总查询。
系统的界面设计如图 \ref{fig:ui} 所示,实现了用户与管理员视角下的数据实时同步。

\section{实验总结}
本实验完成了网咖会员与消费追踪系统的数据库设计与核心功能开发,实现了会员管理、充值、上机消费、消费结算及座位管理等业务逻辑。
通过存储过程封装操作,保证了数据一致性和原子性。
测试结果显示,系统功能运行正常,能够正确处理重复会员注册、座位占用及维修状态等异常情况。

同时,也存在一些不足:对座位状态持续时间的管理不够完善,但这类业务逻辑通常不在数据库层处理,数据库只需支持基础 CRUD 操作即可;
系统尚未对存取效率和大规模数据处理能力进行优化,但考虑到网吧业务量通常不集中,也不是很重要;
较为重要的是,数据库备份和恢复机制的设计需进一步完善,以保障数据安全。
\end{document}